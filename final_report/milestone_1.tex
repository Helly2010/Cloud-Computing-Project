% This is LLNCS.DEM the demonstration file of
% the LaTeX macro package from Springer-Verlag
% for Lecture Notes in Computer Science,
% version 2.3 for LaTeX2e
%
\documentclass{llncs}
%
\usepackage{ngerman}
\usepackage[T1]{fontenc}
\usepackage[utf8]{inputenc}
\usepackage{makeidx}  % allows for indexgeneration
\usepackage{multirow}
\usepackage{rotating}
\usepackage{verbatim}
\usepackage{graphicx}
\usepackage{amssymb}   % AMS-Sonderzeichen
\usepackage{tabularx}  % Für tabularx und newcolumntype
% \usepackage[paper=a4paper,left=30mm,right=30mm,top=30mm,bottom=30mm]{geometry}
\usepackage{color}
\usepackage{ragged2e}
\usepackage{ifpdf}
% \usepackage{titlesec}
\usepackage{xcolor}    % Lieber xcolor als color. Dann klappt auch das listings gut mit den Farben
\usepackage{listings}
\usepackage{upquote}   % Verändert die Ausgabe der einfachen Anführungszeichen innerhalb von verbatim
\usepackage{eurosym}   % Euro-Zeichen: \euro
\usepackage{lastpage}  % \pageref{LastPage} um die Anzahl der Seiten zu erhalten
% hiermit kann man auch umlaute copy-pasten
\usepackage{lmodern}
\selectlanguage{english}
\usepackage{fancyhdr}
\pagestyle{fancy}
%

\ifpdf
\pdfinfo{
   /Author (Wladymir Alexander Brborich Herrera)
   /Author (Vishwaben Pareshbhai Kakadiya)
   /Author (Hellyben Bhaveshkumar Shah)
   /Author (Priyanka Dilipbhai Vadiwala)
   /Author (Heer Rakeshkumar Vankawala)
   /Title  (LowTech GMmBH Techincal Transformation Milestone 1)
   /Subject (Cloud Computing)
   /Keywords (Cloud Computing, Technical Transformation, Migration)
}
\fi

\setlength{\parindent}{0pt}    % Erste Zeile eines Absatzes nicht einrücken
\parskip2ex                    % Absatzabstand
\setlength{\itemsep}{0ex plus0.2ex}
\sloppy                        % Auf jeden Fall die Seitenränder einhalten.

\newcommand{\what}{LowTech GMmBH Techincal Transformation Milestone 1}
\newcommand{\who}{Group 23}
\newcommand{\when}{WiSe 2024-2025}

\renewcommand{\headrulewidth}{0.4pt}
\renewcommand{\footrulewidth}{0.4pt}
\lhead[\when]{\who}
\rhead[\who]{\when}
\chead[]{}
\lfoot[Page \thepage\ of \pageref{LastPage}]{\what}
\rfoot[\what]{Page \thepage\ of \pageref{LastPage}}
\cfoot[]{}
\pagestyle{fancy}


% Hurenkinder und Schusterjungen komplett verbieten.
\clubpenalty = 10000 
\widowpenalty = 10000 
\displaywidowpenalty = 10000
% Diese Begriffe bezeichnen den Makel beim Textsatz, wenn eine Seite mit der ersten Zeile eines Absatzes endet (so genannter Schusterjunge) oder eine neue Seite mit der letzten Zeile eines Absatzes beginnt (so genanntes Hurenkind).


% Wir definieren ein paar Farben
\definecolor{Brown}{cmyk}{0,0.81,1,0.60}
\definecolor{OliveGreen}{cmyk}{0.64,0,0.95,0.40}
\definecolor{CadetBlue}{cmyk}{0.62,0.57,0.23,0}
\definecolor{lightlightgray}{gray}{0.9}
\definecolor{FrankfurtBlue}{HTML}{3333b2}

% Hier fängt das Dokument an!
\begin{document}

%
% \frontmatter          % for the preliminaries
%
% \tableofcontents
%
\mainmatter              % start of the contributions
%
\title{\what}
%
\author{
  Wladymir Alexander Brborich Herrera\\
  \texttt{wladymir.brborich-herrera@stud.fra-uas.de}
  \and\\ 
  Vishwaben Pareshbhai Kakadiya\\
  \texttt{}
  \and\\
  Hellyben Bhaveshkumar Shah\\
  \texttt{}
  \and\\
  Priyanka Dilipbhai Vadiwala\\
  \texttt{}
  \and\\
  Heer Rakeshkumar Vankawala\\
  \texttt{}
}
%
\institute{
  Frankfurt University of Applied Sciences\\
  (1971-2014: Fachhochschule Frankfurt am Main)\\
  Nibelungenplatz 1\\
  D-60318 Frankfurt am Main\\
}

\maketitle              % typeset the title of the contribution

\begin{abstract}
  LowTech GMmBH is a wooden furniture retailer that went public with an online store several years ago. To do so they implemented an on-premise solution. Which not only drives the online store, but all the auxiliary applications i.e. warehouse, customer service, finance and HR software. As demand increases, they are looking to modernize the current infrastructure using a private cloud. This document provides an in depth analysis of of the current infrastructure, including energy consumption metrics, the proposed roadmap and technologies to perform the technical transformation, and finally, a list of potential benefits of the approach, including a simple cost analysis.
  
\end{abstract}

This is where the introduction (the prologue or foreword) comes in. The introduction should also be short and concise. The reader should be prepared for the text that follows. Of course, the introduction should also be formulated in an interesting way.

\section{Overview of the problem}




\section{Objectives of the technological transformation}


\section{Assessment of the current infrastructure}

\subsection{Current traffic and usage}

\subsection{Approximate energy consumption}

\subsection{Scalability, availability and security analysis}




\section{Client Requirements}

\section{Assessment of potential technological components}

\subsection{Hardware}

\subsection{Virtualization technologies}


\subsection{Application components}

\subsection{Platforms}

\subsection{Security components}


\section{Migration to a private-cloud context}

\subsection{Selected technologies}

\subsection{Architecture}

\subsection{Roadmap}

\subsection{Operation considerations}

% ---- Bibliography ----

\begin{thebibliography}{5}
  
  \bibitem{SpringerWeb}
  Information for Authors of Springer Computer Science Proceedings. Springer. 2017\\
  \url{http://www.springer.com/gp/computer-science/lncs/conference-proceedings-guidelines}
  \bibitem{Knuth1974}
  Knuth, Donald~E. \textsl{Computer Programming as an Art}. Communications of the ACM 17 (12). December 1974. S.667-673\\
  \url{https://dl.acm.org/doi/pdf/10.1145/1283920.1283929}
  \bibitem{LaTeXWeb}
  \LaTeX\ -- A document preparation system.
  \url{http://www.latex-project.org}
  \bibitem{LaTeXKopka1}
  Kopka, Helmut. \textsl{\LaTeX, Band 1: Einführung}. Pearson. 2005
  \bibitem{LaTeXKopka2}
  Kopka, Helmut. \textsl{\LaTeX, Band 2: Ergänzungen}. Pearson. 2002
  \bibitem{LaTeXKopka3}
  Kopka, Helmut. \textsl{\LaTeX, Band 3: Erweiterungen}. Pearson. 2002
  \bibitem{LaTeXBegleiter}
  Mittelbach, Frank und Goossens, Michel. \textsl{Der LaTeX-Begleiter}. Pearson. 2005
  \bibitem{LaTeXWissenschaftlich}
  Schlosser, Joachim. \textsl{Wissenschaftliche Arbeiten schreiben mit LaTeX: Leitfaden für Einsteiger}. Mitp-Verlag. 2008
  \bibitem{LaTeXSchnelleinsteiger}
  Willms, Roland. \textsl{\LaTeX: Für Schnelleinsteiger}. Franzis. 2006
\end{thebibliography}
\end{document}
