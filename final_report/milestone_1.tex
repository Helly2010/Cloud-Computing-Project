% This is LLNCS.DEM the demonstration file of
% the LaTeX macro package from Springer-Verlag
% for Lecture Notes in Computer Science,
% version 2.3 for LaTeX2e
%
\documentclass{llncs}
%
\usepackage{ngerman}
\usepackage[T1]{fontenc}
\usepackage[utf8]{inputenc}
\usepackage{makeidx}  % allows for indexgeneration
\usepackage{multirow}
\usepackage{rotating}
\usepackage{verbatim}
\usepackage{graphicx}
\usepackage{amssymb}   % AMS-Sonderzeichen
\usepackage{tabularx}  % Für tabularx und newcolumntype
\usepackage[paper=a4paper,left=25mm,right=25mm,top=25mm,bottom=25mm]{geometry}
\usepackage{array}
\usepackage{makecell}
\usepackage{color}
\usepackage{ragged2e}
\usepackage{ifpdf}
% \usepackage{titlesec}
\usepackage{xcolor}    % Lieber xcolor als color. Dann klappt auch das listings gut mit den Farben
\usepackage{listings}
\usepackage{upquote}   % Verändert die Ausgabe der einfachen Anführungszeichen innerhalb von verbatim
\usepackage{eurosym}   % Euro-Zeichen: \euro
\usepackage{lastpage}  % \pageref{LastPage} um die Anzahl der Seiten zu erhalten
% hiermit kann man auch umlaute copy-pasten
\usepackage{lmodern}
\selectlanguage{english}
\usepackage{fancyhdr}
\usepackage{url}
\pagestyle{fancy}


%

\ifpdf
\pdfinfo{
   /Author (Wladymir Alexander Brborich Herrera)
   /Author (Vishwaben Pareshbhai Kakadiya)
   /Author (Hellyben Bhaveshkumar Shah)
   /Author (Priyanka Dilipbhai Vadiwala)
   /Author (Heer Rakeshkumar Vankawala)
   /Title  (LowTech GMmBH Techincal Transformation Milestone 1)
   /Subject (Cloud Computing)
   /Keywords (Cloud Computing, Technical Transformation, Migration)
}
\fi

\setlength{\parindent}{0pt}    % Erste Zeile eines Absatzes nicht einrücken
\parskip2ex                    % Absatzabstand
\setlength{\itemsep}{0ex plus0.2ex}
\sloppy                        % Auf jeden Fall die Seitenränder einhalten.

\newcommand{\what}{LowTech GMmBH Techincal Transformation Milestone 1}
\newcommand{\who}{Group 23}
\newcommand{\when}{WiSe 2024-2025}

\renewcommand{\headrulewidth}{0.4pt}
\renewcommand{\footrulewidth}{0.4pt}
\lhead[\when]{\who}
\rhead[\who]{\when}
\chead[]{}
\lfoot[Page \thepage\ of \pageref{LastPage}]{\what}
\rfoot[\what]{Page \thepage\ of \pageref{LastPage}}
\cfoot[]{}
\pagestyle{fancy}


% Hurenkinder und Schusterjungen komplett verbieten.
\clubpenalty = 10000 
\widowpenalty = 10000 
\displaywidowpenalty = 10000
% Diese Begriffe bezeichnen den Makel beim Textsatz, wenn eine Seite mit der ersten Zeile eines Absatzes endet (so genannter Schusterjunge) oder eine neue Seite mit der letzten Zeile eines Absatzes beginnt (so genanntes Hurenkind).


% Wir definieren ein paar Farben
\definecolor{Brown}{cmyk}{0,0.81,1,0.60}
\definecolor{OliveGreen}{cmyk}{0.64,0,0.95,0.40}
\definecolor{CadetBlue}{cmyk}{0.62,0.57,0.23,0}
\definecolor{lightlightgray}{gray}{0.9}
\definecolor{FrankfurtBlue}{HTML}{3333b2}

% Hier fängt das Dokument an!
\begin{document}

%
% \frontmatter          % for the preliminaries
%
% \tableofcontents
%
\mainmatter              % start of the contributions
%
\title{\what}
%
\author{
  Wladymir Alexander Brborich Herrera\\
  \texttt{wladymir.brborich-herrera@stud.fra-uas.de}
  \and\\ 
  Vishwaben Pareshbhai Kakadiya\\
  \texttt{vishwaben.kakadiya@stud.fra-uas.de}
  \and\\
  Hellyben Bhaveshkumar Shah (1476905)\\
  \texttt{hellyben.shah@stud.fra-uas.de}
  \and\\
  Priyanka Dilipbhai Vadiwala\\
  \texttt{priyanka.vadiwala@stud.fra-uas.de}
  \and\\
  Heer Rakeshkumar Vankawala\\
  \texttt{heer.vankawala@stud.fra-uas.de}
}
%
\institute{
  Frankfurt University of Applied Sciences\\
  (1971-2014: Fachhochschule Frankfurt am Main)\\
  Nibelungenplatz 1\\
  D-60318 Frankfurt am Main\\
}

\maketitle              % typeset the title of the contribution

\begin{abstract}
In this report, we as consultants from \textit{Awesome Cloud AG} present a technical transformation analysis aimed at modernizing the infrastructure of \textit{LowTech GmbH}, 
a small to medium-sized enterprise specializing in wooden furniture production. 
The analysis includes a critical assessment of the current infrastructure, energy consumption calculation for the existing setup 
followed by a detailed transformation roadmap of future-ready modern infrastructure and explanations of enhancements in scalability, availability, and security compared to current infrastructure. 
This analysis will serve as a foundational step for subsequent project phases, ensuring that \textit{LowTech GmbH} is well-equipped to meet future requirements and challenges.
\end{abstract}

This is where the introduction (the prologue or foreword) comes in. The introduction should also be short and concise. The reader should be prepared for the text that follows. Of course, the introduction should also be formulated in an interesting way.

\section{Overview of the problem}




\section{Objectives of the technological transformation}


\section{Assessment of the current (As-is) infrastructure}
According to NIST definition of cloud computing is given as ``Cloud computing is a model for enabling ubiquitous, convenient, on-demand network access to a shared
pool of configurable computing resources (e.g., networks, servers, storage, applications, and services) that
can be rapidly provisioned and released with minimal management effort or service provider interaction'' \cite{mell2011nist}.

\subsection{Current traffic and usage}

\subsection{Scalability, availability and security analysis}

\subsection{Energy consumption and approximate cost}
Energy consumption calculation for the as-in infrastructure of Low Tech GmbH is as follows : 

\begin{table}[htbp]
\centering
\begin{tabular}{|l|c|c|c|c|c|}
\hline
\textbf{Departments} & \textbf{Server} & \textbf{Client} & \textbf{Laptop} & \textbf{Total Power} & \textbf{Annual Energy}  \\
 & \textbf{ (Qty x Power) } & \textbf{ (Qty x Power) } & \textbf{ (Qty x Power) } & \textbf{ Consumption } & \textbf{ Consumption(KWh) } \\
\hline
Finance & 1 x 1000W & 4 x 500W & - & 3000W & 26,280 \\
\hline
HR & 1 x 1000W & 3 x 500W & - & 2500W & 21,900 \\
\hline
Warehouse & 1 x 1000W & 10 x 500W & - & 6000W & 52,560 \\
\hline
Sales & \makecell{1 x 1000W \\ 1 x 1200W} & - & 10 x 50W & 2700W & 23,652 \\
\hline
Operations & 1 x 1200W & - & 4 x 50W & 1400W & 12,264\\
\hline
Customer Service & - & - & 5 x 100W & 500W & 4,380 \\
\hline
Webshop & 1 x 1200W & - & - & 1200W & 10,512\\
\hline
\end{tabular}
\caption{Power Consumption by Department and Device Type}
\label{tab:power_consumption}
\end{table}

\textbf{Total Energy Consumption (Annual)} : 151,548 KWh (151.548 MWh)

According to Eurostat published data of electricity prices for non-household consumers \cite{eurostat2023}, Low Tech GmbH falls under the annual energy consumption band 
`IB (20 MWh to 499 MWh)' with energy price 0.3244 \EUR{} per KWh.


\textbf{Total Cost for Energy Consumption (Annual)} : 151,548 KWh x 0.3244 \EUR{} = 49,162.17 \EUR{}







\section{Client Requirements}

\section{Assessment of potential technological components}

\subsection{Hardware}

\subsection{Virtualization technologies}


\subsection{Application components}

\subsection{Platforms}

\subsection{Security components}


\section{Migration to a private-cloud context}

\subsection{Selected technologies}

\subsection{Architecture}

\subsection{Roadmap}

\subsection{Operation considerations}

% ---- Bibliography ----

\begin{thebibliography}{5}
  
  \bibitem{eurostat2023}
Eurostat. (2023). \textit{Electricity prices for non-household consumers - bi-annual data (from 2007 onwards)}. Retrieved November 16, 2023, from \url{https://ec.europa.eu/eurostat/databrowser/view/nrg_pc_205__custom_13581723/default/table?lang=en}

\bibitem{mell2011nist}
Mell, P. and Grance, T. (2011).
\emph{The NIST definition of cloud computing}.
National Institute of Standards and Technology, Special Publication 800-145, Gaithersburg, MD.
\url{https://nvlpubs.nist.gov/nistpubs/Legacy/SP/nistspecialpublication800-145.pdf} 

\end{thebibliography}
\end{document}
