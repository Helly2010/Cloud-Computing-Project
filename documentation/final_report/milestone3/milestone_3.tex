% This is LLNCS.DEM the demonstration file of
% the LaTeX macro package from Springer-Verlag
% for Lecture Notes in Computer Science,
% version 2.3 for LaTeX2e



\documentclass{llncs}


\usepackage{ngerman}
\usepackage[T1]{fontenc}
\usepackage[utf8]{inputenc}
\usepackage{makeidx}  % allows for indexgeneration
\usepackage{multirow}
\usepackage{rotating}
\usepackage{verbatim}
\usepackage{graphicx}
\usepackage{float}
\usepackage{graphicx}  % For \resizebox
\usepackage{amssymb}   % AMS-Sonderzeichen
\usepackage{tabularx}  % Für tabularx und newcolumntype
\usepackage[paper=a4paper,left=25mm,right=25mm,top=25mm,bottom=25mm]{geometry}
\usepackage{array}
\usepackage{makecell}
\usepackage{color}
\usepackage{ragged2e}
\usepackage{longtable}
\usepackage{ifpdf}
% \usepackage{titlesec}
\usepackage{xcolor}    % Lieber xcolor als color. Dann klappt auch das listings gut mit den Farben
\usepackage{listings}
\usepackage{upquote}   % Verändert die Ausgabe der einfachen Anführungszeichen innerhalb von verbatim
\usepackage{eurosym}   % Euro-Zeichen: \euro
\usepackage{lastpage}  % \pageref{LastPage} um die Anzahl der Seiten zu erhalten
% hiermit kann man auch umlaute copy-pasten
\usepackage{lmodern}
\selectlanguage{english}
\usepackage{fancyhdr}
\usepackage{url}
\usepackage{caption}
\usepackage{float}
\setlength{\abovecaptionskip}{0pt} % Reduce space above the caption
\setlength{\belowcaptionskip}{0pt} % Space between caption and table
\captionsetup[table]{skip=5pt} % Adjust the value of '10pt' as needed
\pagestyle{fancy}
% Reduce space between float and surrounding text
\setlength{\textfloatsep}{0pt}
\setlength{\floatsep}{0pt}
\setlength{\intextsep}{0pt}

% Enable subsubsection numbering
\setcounter{secnumdepth}{3}

%

\ifpdf
\pdfinfo{
 /Author (Wladymir Alexander Brborich Herrera)
 /Author (Vishwaben Pareshbhai Kakadiya)
 /Author (Hellyben Bhaveshkumar Shah)
 /Author (Heer Rakeshkumar Vankawala)
 /Author (Priyanka Dilipbhai Vadiwala)
 /Title  (LowTech GMBH Techincal Transformation Milestone 3)
 /Subject (Cloud Computing)
 /Keywords (Cloud Computing, Technical Transformation, Migration)
}
\fi

\setlength{\parindent}{0pt}    % Erste Zeile eines Absatzes nicht einrücken
\parskip2ex                    % Absatzabstand
\setlength{\itemsep}{0ex plus0.2ex}
\sloppy                        % Auf jeden Fall die Seitenränder einhalten.

\newcommand{\what}{Milestone 3: Practical Implementation of LowTech Gmbh Webshop in CSP Platform}
\newcommand{\who}{Group 23}
\newcommand{\when}{WiSe 2024-2025}

\renewcommand{\headrulewidth}{0.4pt}
\renewcommand{\footrulewidth}{0.4pt}
\lhead[\when]{\who}
\rhead[\who]{\when}
\chead[]{}
\lfoot[Page \thepage\ of \pageref{LastPage}]{\what}
\rfoot[\what]{Page \thepage\ of \pageref{LastPage}}
\cfoot[]{}
\pagestyle{fancy}


% Hurenkinder und Schusterjungen komplett verbieten.
\clubpenalty = 10000 
\widowpenalty = 10000 
\displaywidowpenalty = 10000
% Diese Begriffe bezeichnen den Makel beim Textsatz, wenn eine Seite mit der ersten Zeile eines Absatzes endet (so genannter Schusterjunge) oder eine neue Seite mit der letzten Zeile eines Absatzes beginnt (so genanntes Hurenkind).


% Wir definieren ein paar Farben
\definecolor{Brown}{cmyk}{0,0.81,1,0.60}
\definecolor{OliveGreen}{cmyk}{0.64,0,0.95,0.40}
\definecolor{CadetBlue}{cmyk}{0.62,0.57,0.23,0}
\definecolor{lightlightgray}{gray}{0.9}
\definecolor{FrankfurtBlue}{HTML}{3333b2}

% Hier fängt das Dokument an!
\begin{document}

%
% \frontmatter          % for the preliminaries
%
% \tableofcontents
%
\mainmatter              % start of the contributions
%
\title{\what}
%
\author{
    Wladymir Alexander Brborich Herrera (1437876)\\
    \texttt{wladymir.brborich-herrera@stud.fra-uas.de}
    \and\\
    Vishwaben Pareshbhai Kakadiya (1471845)\\
    \texttt{vishwaben.kakadiya@stud.fra-uas.de}
    \and\\
    Hellyben Bhaveshkumar Shah (1476905)\\
    \texttt{hellyben.shah@stud.fra-uas.de}
    \and\\
    Heer Rakeshkumar Vankawala (1449039)
    \\
    \texttt{heer.vankawala@stud.fra-uas.de}
    \and\\
    Priyanka Dilipbhai Vadiwala (1481466)\\
    \texttt{priyanka.vadiwala@stud.fra-uas.de}
}
%
\institute{
    Frankfurt University of Applied Sciences\\
    (1971-2014: Fachhochschule Frankfurt am Main)\\
    Nibelungenplatz 1\\
    D-60318 Frankfurt am Main\\
}

\maketitle              % typeset the title of the contribution


\section*{Abstract}

LowTech GmbH Cloud Transformation Project, building upon the previous analyses and migration strategies. In this phase,  
the focus is on optimizing cloud operations, performance monitoring, and cost efficiency after the transition to Microsoft  
Azure. The objective is to ensure system reliability, security, and scalability while fine-tuning the deployed infrastructure.  
The report outlines key post-migration strategies, including performance assessment, security enhancements, and cost analysis.  
It introduces automation techniques using tools like Terraform and Ansible for infrastructure management and GitHub Actions for  
continuous integration and deployment (CI/CD). Special attention is given to monitoring solutions like Azure Monitor and Prometheus  
to track system performance and detect potential issues in real time.  Furthermore, we evaluate cloud cost management techniques by analyzing usage patterns and identifying areas for optimization.  
The report also discusses future scalability strategies, ensuring that LowTech GmbH is well-equipped to handle growing business demands.  
This phase serves as the foundation for long-term cloud sustainability, enabling the company to leverage cloud-native solutions efficiently  
while maintaining operational resilience.  

\section{Introduction}
\subsection{Overview of the Project}
LowTech GmbH, a medium-sized enterprise specializing in wooden furniture production, is modernizing its IT infrastructure as part of a comprehensive cloud transformation. Initially relying on traditional on-premises systems, the company faced challenges in scalability, security, and operational efficiency.

The transformation began with a thorough assessment of the existing infrastructure, which revealed the following key challenges:
\begin{itemize}
    \item Limited scalability due to fixed hardware constraints.
    \item High operational costs and energy consumption of legacy systems.
    \item Outdated security measures, including basic firewall protection.
    \item Lack of automation, requiring manual interventions for maintenance and scaling.
\end{itemize}

To address these challenges, a private cloud migration strategy was adopted, focusing on hyper-converged infrastructure (HCI) and virtualization using Proxmox and Ansible. The aim was to improve scalability, security, and cost-efficiency while ensuring minimal downtime and business continuity.
The company transitioned its IT ecosystem to Microsoft Azure, leveraging a hybrid cloud architecture that integrates Azure IaaS, PaaS, and SaaS solutions. Key steps in the migration included:
\begin{itemize}
    \item Migrating critical business applications such as Finance, HR, Operations, Webshop, and Warehouse.
    \item Implementing DevOps pipelines using GitHub Actions for automated deployments.
    \item Enhancing monitoring capabilities through Azure Monitor and Prometheus.
    \item Optimizing cloud costs through resource scaling and usage analysis.
\end{itemize}

This strategic approach positions LowTech GmbH to adapt to business growth, enhance security, and optimize costs, all while leveraging cloud-native services for long-term sustainability.


\subsection{Objectives of the Cloud Implementation of Webshop}

The Webshop is a critical component for LowTech GmbH, serving as the company’s primary sales platform. The goal of its cloud implementation is to enhance performance, scalability, and security, ensuring a seamless user experience.

Key objectives for the Webshop's cloud-based deployment include:

\begin{itemize}
    \item \textbf{Scalability and Performance Optimization:}  
    The Webshop is deployed on Azure App Service with auto-scaling capabilities, enabling efficient traffic handling during peak periods. A load balancer ensures even traffic distribution across multiple instances.
    \bigskip % Adds space before the next section
    \item \textbf{High Availability and Reliability:}  
    Azure Virtual Machine Scale Sets provide fault tolerance with automatic failover. Azure Blob Storage is used to securely store digital assets with high availability.
    \bigskip % Adds space before the next section
    \item \textbf{Security and Compliance:}  
    The Webshop integrates with Microsoft Entra ID for user authentication and Azure Security Center for enhanced threat protection. Encryption and Role-Based Access Control (RBAC) are employed to safeguard sensitive customer data.
    \bigskip % Adds space before the next section
    \item \textbf{Continuous Deployment and DevOps Automation:}  
    A CI/CD pipeline powered by GitHub Actions automates code deployments, improving deployment speed and reducing manual intervention.
    \bigskip % Adds space before the next section
    \item \textbf{Cost Efficiency and Resource Optimization:}  
    The dynamic allocation of resources optimizes compute and storage usage, reducing operational expenses. Azure’s pay-as-you-go model aids in cost forecasting and budget management.
    \bigskip % Adds space before the next section
    \item \textbf{Future-Proofing and Cloud-Native Development:}  
    The Webshop follows cloud-native best practices, utilizing Docker containers for portability and Azure Kubernetes Service (AKS) for container orchestration, positioning the Webshop for seamless integration with future cloud services.
\end{itemize}

These objectives ensure that the Webshop is scalable, secure, and cost-effective, delivering a fast and reliable shopping experience while adapting to the evolving needs of the business.

\section{Application Design}
\subsection{Architectural Overview}
Detailed description of the three-tier structure with CSP service mapping
\subsubsection{Presentation-Tier (Frontend) - User Interface (UI)}
\paragraph{Technology Stack}
\begin{itemize}
    \item Frontend Framework : React.js (JavaScript)
    \item State Management : React Context API
    \item Communication with Backend : FastAPI
    \item Hosting \& Deployment : Azure Static Web Apps
 
\end{itemize}

\paragraph{component-based architecture}
\begin{enumerate}
    \item Navigation \& Routing 
    Users can easily navigate between different sections of application, such as viewing product lists, accessing product details, and managing their shopping cart. React Router enables seamless client-side navigation, keeping your app as a single-page application (SPA).
    \begin{itemize}
        \item React Router Setup 
        \begin{itemize}
            \item React Router (react-router-dom) to handle the routing of different components, allowing users to navigate through pages like the homepage (/), product detail pages (/product/:id), and potentially a shopping cart or checkout page.
            \item For product detail pages, you're using dynamic routes with product/:id to fetch and display specific product data based on the product’s unique id. For instance:
            \item When a user clicks on a product in the catalog, the URL changes to something like /product/123, and the ProductDetail component is rendered with data for product 123.
            \item This is achieved using useNavigate and useParams hooks provided by React Router to capture the dynamic part of the URL.
        \end{itemize}
        
        \item Navigation Links
        \begin{itemize}
            \item On the homepage (Home component), displaying a list of products. Each product has an image and name that users can click. When clicked, the app navigates to the product detail page using the navigate(/product/\${prod.id}) function.\\
        \end{itemize}
    \end{itemize}
    \item API Communication 
    \begin{itemize}
        \item API communication in app is responsible for sending and receiving data from the backend server. This includes fetching product data to populate catalog, handling cart actions (adding or removing items), and processing payments during checkout.
        \item The app communicates with the backend to update the cart whenever an item is added or removed. When a user clicks ``Add to Cart'' or ``Remove from Cart,'' an API call might be made to update the user's cart data on the server. This interaction is handled by \texttt{CartContext}, which uses \texttt{dispatch} to update the cart state and might also send a request to the backend to keep the cart in sync.
        \item For payment, we use PayPal to securely handle credit card transactions. The app makes API calls to PayPal's backend to process the payment once the user submits their payment details.\\
    \end{itemize}
    \item Data Fetching \\
    Data fetching refers to the process of retrieving data from external sources (your backend API) to populate app with dynamic information, such as product details, cart contents, and order history.
    \begin{itemize}
        \item The product data, which includes the list of products with details such as images, names, prices, and categories, is fetched when the homepage or product catalog page loads. This is done using useEffect to trigger an API call and update the state with the fetched data.
    \end{itemize}
\end{enumerate}
\paragraph{Key Features of the UI}
    \begin{enumerate}
        \item UI Components \& Design \\
        The UI components in your app are designed to provide an intuitive and engaging shopping experience. Each component serves a specific role, ensuring modularity, reusability, and maintainability.
        \begin{itemize}
            \item Key Features 
            \begin{itemize}
                \item Reusable Components: Components like SingleProduct.js, ProductDetail.js, and Cart.js ensure a structured and modular UI. 
                \item Bootstrap Integration: react-bootstrap is used to style UI components such as Card, Button, and Container, ensuring a consistent and responsive design.
                \item Dark \& Light Theme Support: A theme context (ThemeContextProvider) dynamically adjusts UI styles based on user preference.\\
            \end{itemize}
        \end{itemize}

        \item Product Display \& Interaction \\
        The UI effectively displays products with details, allowing users to browse, select, and view product specifications before making a purchase.
            \begin{itemize}
                \item Key Features 
                \begin{itemize}
                    \item Product Catalog: Displays a grid of available products with images, names, and prices.
                    \item Product Cards (SingleProduct.js):
                    \begin{itemize}
                        \item Clickable product images navigate to the product details page.
                        \item Shows product information such as name, category, description, and price.
                        \item ``Add to Cart'' and ''Remove from Cart'' buttons enable quick cart management.
                    \end{itemize}
                    \item Product Details Page (ProductDetail.js):
                    \begin{itemize}
                        \item Displays full product details, including stock availability.
                        \item Users can add/remove items from the cart.\\
    
                    \end{itemize}
                \end{itemize}
            \end{itemize}

        \item Shopping Cart UI \& Checkout\\
        The cart and checkout sections provide an intuitive way for users to review their selected products, manage quantities, and complete purchases.
            \begin{itemize}
                \item Key Features 
                \begin{itemize}
                    \item Cart UI (Cart.js)
                    \begin{itemize}
                        \item Displays a list of selected products with prices and a ''Remove from Cart'' button.
                        \item Updates total price dynamically based on the cart's contents.
                        \item Navigates users to checkout when ready.
                    \end{itemize}
                    \item Checkout UI (CheckoutForm.js)
                    \begin{itemize}
                        \item Collects user details (name, email).
                        \item Integrates Stripe and PayPal for secure payment processing.
                        \item Uses emailjs to send order confirmation emails.
                        \item Displays success messages with reference numbers for placed orders.\\
                    \end{itemize}
                \end{itemize}
            \end{itemize}

        \item Notifications\\
        Notifications is crucial to enhancing the shopping experience. The app uses toast notifications and error messages to keep users informed.
            \begin{itemize}
                \item Key Features 
                \begin{itemize}
                    \item Toast Notifications (react-toastify)
                    \begin{itemize}
                        \item Displays success messages when products are added/removed from the cart.
                        \item Shows error messages if something goes wrong (e.g., out-of-stock products, payment failures).\\
                    \end{itemize}
                \end{itemize}
            \end{itemize}
    \end{enumerate}
    
\subsubsection{Application-Tier (Backend) - Business Logic}
\paragraph{Technology Stack}
    \begin{itemize}
        \item FastAPI
        \item SQLAlchemy with PostgreSQL
        \item Asyncio
    \end{itemize}

\paragraph{Product Management} \leavevmode

The Product Management module handles the lifecycle of products in the system. This includes operations for adding new products, updating product details, and retrieving product information. The core business logic ensures that products are categorized correctly, their stock levels are managed, and prices are updated as needed.
    \begin{itemize}
        \item Key Operations:
        \begin{itemize}
        \item Add, update, and delete products.
        \item Manage product details such as descriptions, prices, and categories.
        \item Track stock availability and reorder levels.
        \item Calculate the public unit price and manage suppliers' pricing.
        \end{itemize}   
    \end{itemize}

\paragraph{Order Management} \leavevmode

The Order Management module manages customer orders, from order creation to order completion. It tracks the order status and ensures that orders are processed correctly, including payment validation, stock management, and shipping.
\begin{itemize}
    \item Key Operations:
    \begin{itemize}
    \item Create and update orders with customer and product information.
    \item Monitor order statuses (e.g., processing, dispatched, delivered).
    \item Validate inventory and ensure product availability during order processing.
    \item Handle refunds, cancellations, and partial shipments.
    \end{itemize}   
\end{itemize}

\paragraph{Payment Processing} \leavevmode

The Payment Processing module integrates with third-party payment gateways (e.g., Stripe) to securely process customer payments. This module validates payment details, checks for fraud, and ensures payment is successfully processed before confirming orders.
\begin{itemize}
    \item Key Operations:
    \begin{itemize}
    \item Process payments securely through APIs like Stripe and PayPal.
    \item Validate payment information (e.g., credit card details).
    \item Handle refunds, cancellations, and partial shipments.
    \end{itemize}   
\end{itemize}

\paragraph{Inventery Management} \leavevmode

The Inventory Management module tracks the stock levels of all products, ensuring that inventory is updated in real-time based on orders placed and products received from suppliers. It also tracks reorder levels and alerts the system to restock low inventory.
\begin{itemize}
    \item Key Operations:
    \begin{itemize}   
        \item Monitor and update product stock levels after each sale.
        \item Track the supplier’s stock and delivery lead times.
        \item Generate reports on product availability, low-stock items, and reorder recommendations.
    \end{itemize}   
\end{itemize}
    


\paragraph{Email Notification} \leavevmode

The Email Notification module sends emails to customers and administrators for various events in the system, such as order confirmations, shipment tracking updates, and payment status notifications. It integrates with FastMail to send HTML-formatted emails with dynamic content.
\begin{itemize}
    \item Key Operations:
    \begin{itemize}   
        \item Send order confirmation emails to customers.
        \item Notify customers of order status updates, such as dispatched or delivered.
        \item Alert administrators of low stock levels and other system alerts.
        \item Provide a user-friendly email template system for various types of notifications.
    \end{itemize}   
\end{itemize}

\subsubsection{Data-Tier (Database) - Databases}
\paragraph{Technology Stack}
\begin{itemize}   
    \item PostgreSQL
    \item SQLAlchemy ORM
    \item Asyncio \& Asynchronous SQLAlchemy
    \item Alembic
    \item Azure Database for PostgreSQL
\end{itemize}   
\paragraph{Data bases}
\begin{enumerate}
    \item Product Data\\
    \begin{itemize}
        \item Product
        \begin{itemize}
            \item Stores the core product information and links to categories, suppliers, and inventory.
            \item Key Attributes:
            \begin{itemize}
                \item id: Unique product identifier (Primary Key).
                \item name: Product name.
                \item description: Product description.
                \item category\_id: Foreign Key linking to the Categories table.
                \item supplier\_id: Foreign Key linking to the Suppliers table.
                \item stock\_id: Links to the Stock table for inventory tracking.
                \item public\_unit\_price: Price displayed to customers.
                \item supplier\_unit\_price: Price from the supplier (for internal calculations).
                \item ean\_code: Unique product code for identification.
                \item reorder\_level: Threshold for restocking.
                \item img\_link: URL to the product image.
                \item extra\_info: JSON field to store additional product attributes. \\
            \end{itemize}
        \end{itemize}
        \item Catagories
        \begin{itemize}
        \item This table categorizes products for better organization and searchability.
        \item Key Attributes:
        \begin{itemize}
            \item id: Unique category identifier (Primary Key).
            \item name: Product name.
            \item description: Category description.
            \item extra\_info: JSON field for additional metadata.\\
        \end{itemize}
    \end{itemize}
    \end{itemize} 


    \item Order Data\\
    \begin{itemize}
        \item Order 
        \begin{itemize}
            \item This table records general order details and customer information.
            \item Key Attributes:
            \begin{itemize}
                \item id: Unique order identifier (Primary Key).
                \item customer\_name: Name of the customer.
                \item customer\_email: Customer's email address for notifications.
                \item customer\_phone: Contact number.
                \item order\_total: Total value of the order.
                \item status: Enum representing the order state (e.g., ''active'', ''cancelled'').
                \item tracking\_status: Enum tracking delivery progress (e.g., ''dispatched'', ''delivered'').
                \item payment\_method: JSON field storing payment details (e.g., card type).
                \item customer\_shipping\_info: JSON field for customer delivery address.
                \item created\_at: Timestamp when the order was created.
                \item updated\_at: Timestamp when the order was last updated.\\
            \end{itemize}
        \end{itemize}
        \item Order Details
        \begin{itemize}
            \item This table provides itemized details of each product within an order.
            \item Key Attributes:
            \begin{itemize}
                \item id: Unique identifier for each order detail record (Primary Key).
                \item order\_id: Foreign Key linking to the Orders table.
                \item product\_id: Foreign Key linking to the Products table.
                \item quantity: Number of units of the product ordered.
                \item product\_price: Unit price at the time of the order.
                \item subtotal: Calculated subtotal for the item.\\
                
            \end{itemize}
        \end{itemize}
     \end{itemize}

    \item Inventory Data\\
    \begin{itemize}
        \item Stocks 
        \begin{itemize}
            \item Monitors the available quantity of each product and manages restocking.
            \item Key Attributes:
            \begin{itemize}
                \item id: Unique stock identifier (Primary Key).
                \item quantity: Number of available units for a product.
                \item updated\_at: Timestamp of the last stock update.
                \item created\_at: Timestamp when the stock record was created.\\
                
            \end{itemize}
        \end{itemize}
        \item suppliers
        \begin{itemize}
            \item Manages supplier-related information and links to products for procurement.
            \item Key Attributes:
            \begin{itemize}
                \item id: Unique supplier identifier (Primary Key).
                \item name: Supplier name.
                \item address: Supplier’s physical address.
                \item phone: Contact number.
                \item email: Contact email address.\\
                
            \end{itemize}
        \end{itemize}
    \end{itemize}
    \item  Database Migrations with Alembic \\
    \begin{itemize}
        \item It allows us to track database changes, apply updates to production, and roll back changes when needed.
        \item Key Operations with Alembic:
        \begin{itemize}
            \item \textit{Schema Migration:} Automatically generates migration scripts to apply changes to the database (e.g., adding new fields to a table).
            \item \textit{Version Control:} Each migration is versioned, allowing us to track and audit changes.
            \item \textit{Rollback Support:} Provides the ability to downgrade to previous schema versions if issues arise.\\  
        \end{itemize}
    \end{itemize}
    \end{enumerate}

\subsection{System Diagrams}

\section{Implementation Process}
\subsection{Cloud Environment Setup}
Step-by-step account configuration and resource provisioning

\subsection{Service Integration}
\begin{itemize}
    \item Azure Load Balancer configuration
    \item Database replication setup
    \item Blob storage integration patterns
\end{itemize}

\subsection{Development Challenges}
\begin{itemize}
    \item State management in scaled environments
    \item Database connection pooling
    \item CSP-specific limitations encountered
\end{itemize}

\section{Operational Characteristics}
\subsection{Performance Metrics}

\subsubsection{Functional Test Cases}

\begin{longtable}{|c|p{4.5cm}|p{4.5cm}|c|}
\hline
\textbf{Test Case ID} & \textbf{Test Scenario} & \textbf{Expected Outcome} & \textbf{Status (Pass/Fail)} \\ \hline
\endfirsthead
\hline
\textbf{Test Case ID} & \textbf{Test Scenario} & \textbf{Expected Outcome} & \textbf{Status (Pass/Fail)} \\ \hline
\endhead
TC-001 & Load homepage and verify product listing & Homepage loads with products displayed correctly. & \\ \hline
TC-002 & Apply price filter (Ascending/Descending) & Products reorder correctly based on selected price. & \\ \hline
TC-003 & Apply "Out of Stock" filter & Only out-of-stock items are displayed. & \\ \hline
TC-004 & Apply "Fast Delivery" filter & Only products eligible for fast delivery show up. & \\ \hline
TC-005 & Filter by category & Products are filtered correctly by selected category. & \\ \hline
TC-006 & Search product by name or category & Products matching search are shown correctly. & \\ \hline
TC-007 & Clear filter functionality & All filters are removed, showing the full product list. & \\ \hline
TC-008 & Product Detail Page - Click Product & Clicking a product opens its detailed page. & \\ \hline
TC-009 & Product Detail Page - Load Product Details & Product details (name, description, price) are displayed. & \\ \hline
TC-010 & Add a product to cart & Product appears in cart with correct details. & \\ \hline
TC-011 & Increase product quantity in cart & Quantity updates and is reflected in the cart. & \\ \hline
TC-012 & Remove product from cart & Product is removed from the cart immediately. & \\ \hline
TC-013 & Proceed to checkout & Checkout page loads with the correct order summary. & \\ \hline
TC-014 & Select payment method - Stripe & Stripe payment option is selected and processed. & \\ \hline
TC-015 & Select payment method - PayPal & PayPal payment option is selected and processed. & \\ \hline
TC-016 & Complete order processing & Order confirmation message is displayed. & \\ \hline
TC-017 & Order confirmation email is received & Email is sent after order is placed. & \\ \hline
TC-018 & Shipment notification email is received & Email is sent when the order is shipped. & \\ \hline
TC-019 & Toggle Dark/Light Theme & Application switches between themes successfully. & \\ \hline
\end{longtable}

\subsection{Security Considerations}
\begin{itemize}
    \item Network security groups configuration
    \item Database encryption implementation
    \item Access control mechanisms
\end{itemize}

\section{Critical Analysis}
\subsection{Cloud Service Evaluation}
Cost-benefit analysis of selected Azure services
\subsection{Architectural Decisions}
Trade-off discussion between containerized vs serverless approaches

\section{Repository Documentation}
\subsection{GitHub Structure}


\begin{enumerate}
    \item \textbf{Branching Strategy:} \newline 
     The project follows a simple main branch strategy where all contributors work directly on the main branch.  
    This ensures seamless integration without managing multiple branches. All changes should be committed with meaningful messages.  
    Code should be reviewed and tested before pushing to the main branch.
\end{enumerate}

\begin{enumerate}
    \setcounter{enumi}{1}
    \item \textbf{Continuous Integration/Continuous Deployment:}  
    \begin{itemize}
        \item \textbf{Continuous Integration (CI):}  
        
        Every code commit triggers an automated build and testing process. This includes:
        \begin{itemize}
            \item Running unit tests to validate individual components.
            \item Performing integration tests to ensure compatibility between different modules.
            \item Static code analysis for linting and security vulnerabilities.
        \end{itemize}

        \bigskip % Adds space before the next section
        \item \textbf{Continuous Deployment (CD):}  
        
        Once the CI stage passes successfully, the application is automatically deployed to the appropriate environment. This process includes:
        \begin{itemize}
            \item Deploying to a staging environment for final testing.
            \item Running automated acceptance tests before production deployment.
            \item Deploying to production with rollback mechanisms in case of failure.
        \end{itemize}

        \bigskip % Adds space before the next section

        \item \textbf{Documentation Standards}
    \end{itemize}
\end{enumerate}


\subsection{Contribution Tracking}
Commit history analysis and individual contribution breakdown

\section{Conclusion}
\subsection{Project Outcomes}
Summary of achieved objectives and demo capabilities

\subsection{Future Enhancements}
Potential improvements for production readiness

\begin{thebibliography}{9}

\end{thebibliography}
\end{document}

