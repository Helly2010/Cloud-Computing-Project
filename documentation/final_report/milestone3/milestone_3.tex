% This is LLNCS.DEM the demonstration file of
% the LaTeX macro package from Springer-Verlag
% for Lecture Notes in Computer Science,
% version 2.3 for LaTeX2e



\documentclass{llncs}


\usepackage{ngerman}
\usepackage[T1]{fontenc}
\usepackage[utf8]{inputenc}
\usepackage{makeidx}  % allows for indexgeneration
\usepackage{multirow}
\usepackage{rotating}
\usepackage{verbatim}
\usepackage{graphicx}
\usepackage{float}
\usepackage{graphicx}  % For \resizebox
\usepackage{amssymb}   % AMS-Sonderzeichen
\usepackage{tabularx}  % Für tabularx und newcolumntype
\usepackage[paper=a4paper,left=25mm,right=25mm,top=25mm,bottom=25mm]{geometry}
\usepackage{array}
\usepackage{makecell}
\usepackage{color}
\usepackage{ragged2e}
\usepackage{longtable}
\usepackage{ifpdf}
% \usepackage{titlesec}
\usepackage{xcolor}    % Lieber xcolor als color. Dann klappt auch das listings gut mit den Farben
\usepackage{listings}
\usepackage{upquote}   % Verändert die Ausgabe der einfachen Anführungszeichen innerhalb von verbatim
\usepackage{eurosym}   % Euro-Zeichen: \euro
\usepackage{lastpage}  % \pageref{LastPage} um die Anzahl der Seiten zu erhalten
% hiermit kann man auch umlaute copy-pasten
\usepackage{lmodern}
\selectlanguage{english}
\usepackage{fancyhdr}
\usepackage{url}
\usepackage{caption}
\usepackage{float}

\setlength{\abovecaptionskip}{0pt} % Reduce space above the caption
\setlength{\belowcaptionskip}{0pt} % Space between caption and table
\captionsetup[table]{skip=5pt} % Adjust the value of '10pt' as needed
\pagestyle{fancy}
% Reduce space between float and surrounding text
\setlength{\textfloatsep}{0pt}
\setlength{\floatsep}{0pt}
\setlength{\intextsep}{0pt}

% Enable subsubsection numbering
\setcounter{secnumdepth}{3}

%

\ifpdf
\pdfinfo{
 /Author (Wladymir Alexander Brborich Herrera)
 /Author (Vishwaben Pareshbhai Kakadiya)
 /Author (Hellyben Bhaveshkumar Shah)
 /Author (Heer Rakeshkumar Vankawala)
 /Author (Priyanka Dilipbhai Vadiwala)
 /Title  (LowTech GMBH Techincal Transformation Milestone 3)
 /Subject (Cloud Computing)
 /Keywords (Cloud Computing, Technical Transformation, Migration)
}
\fi

\setlength{\parindent}{0pt}    % Erste Zeile eines Absatzes nicht einrücken
\parskip2ex                    % Absatzabstand
\setlength{\itemsep}{0ex plus0.2ex}
\sloppy                        % Auf jeden Fall die Seitenränder einhalten.

\newcommand{\what}{Milestone 3: Practical Implementation of LowTech Gmbh Webshop in CSP Platform}
\newcommand{\who}{Group 23}
\newcommand{\when}{WiSe 2024-2025}

\renewcommand{\headrulewidth}{0.4pt}
\renewcommand{\footrulewidth}{0.4pt}
\lhead[\when]{\who}
\rhead[\who]{\when}
\chead[]{}
\lfoot[Page \thepage\ of \pageref{LastPage}]{\what}
\rfoot[\what]{Page \thepage\ of \pageref{LastPage}}
\cfoot[]{}
\pagestyle{fancy}


% Hurenkinder und Schusterjungen komplett verbieten.
\clubpenalty = 10000 
\widowpenalty = 10000 
\displaywidowpenalty = 10000
% Diese Begriffe bezeichnen den Makel beim Textsatz, wenn eine Seite mit der ersten Zeile eines Absatzes endet (so genannter Schusterjunge) oder eine neue Seite mit der letzten Zeile eines Absatzes beginnt (so genanntes Hurenkind).


% Wir definieren ein paar Farben
\definecolor{Brown}{cmyk}{0,0.81,1,0.60}
\definecolor{OliveGreen}{cmyk}{0.64,0,0.95,0.40}
\definecolor{CadetBlue}{cmyk}{0.62,0.57,0.23,0}
\definecolor{lightlightgray}{gray}{0.9}
\definecolor{FrankfurtBlue}{HTML}{3333b2}

% Hier fängt das Dokument an!
\begin{document}

%
% \frontmatter          % for the preliminaries
%
% \tableofcontents
%
\mainmatter              % start of the contributions
%
\title{\what}
%
\author{
    Wladymir Alexander Brborich Herrera (1437876)\\
    \texttt{wladymir.brborich-herrera@stud.fra-uas.de}
    \and\\
    Vishwaben Pareshbhai Kakadiya (1471845)\\
    \texttt{vishwaben.kakadiya@stud.fra-uas.de}
    \and\\
    Hellyben Bhaveshkumar Shah (1476905)\\
    \texttt{hellyben.shah@stud.fra-uas.de}
    \and\\
    Heer Rakeshkumar Vankawala (1449039)
    \\
    \texttt{heer.vankawala@stud.fra-uas.de}
    \and\\
    Priyanka Dilipbhai Vadiwala (1481466)\\
    \texttt{priyanka.vadiwala@stud.fra-uas.de}
}
%
\institute{
    Frankfurt University of Applied Sciences\\
    (1971-2014: Fachhochschule Frankfurt am Main)\\
    Nibelungenplatz 1\\
    D-60318 Frankfurt am Main\\
}

\maketitle              % typeset the title of the contribution


\begin{abstract}

    LowTech GmbH Cloud Transformation Project, building upon the previous analyses and migration strategies. In this phase, 
    the focus is on optimizing cloud operations, performance monitoring, and cost efficiency after the transition to Microsoft
    Azure. The objective is to ensure the system reliability, security, and scalability while fine-tuning the deployed infrastructure.
    The report outlines key post-migration strategies, including performance assessment, security enhancements, and cost analysis. 
    It introduce automation techniques using tools like Terraform and Ansible for infrastructure management and GitHub Actions for 
    continuous integration and deployment (CI/CD). Special attention is given to monitoring solutions like Azure Monitor and Prometheus
    to track system performance and detect potential issues in real time. Furthermore, we evaluate cloud cost management techniques by 
    analyzing usage patterns and identifying areas for optimization.The report also discusses future scalability strategies,
    ensuring that LowTech GmbH is well-equipped to handle growing business demands.This phase serves as the foundation for long-term 
    cloud sustainability, enabling the company to leverage cloud-native solutions efficiently while maintaining operational resilience.

\end{abstract}

\section{Introduction}
\subsection{Overview of the Project}
Brief recap of previous milestones and progression to current implementation phase

\subsection{Objectives of the Cloud Implementation of Webshop}


\section{Application Design}
\subsection{Architectural Overview}
Detailed description of the three-tier structure with CSP service mapping
\subsubsection{Presentation-Tier (Frontend) - User Interface (UI)}
\paragraph{Technology Stack}
\begin{itemize}
    \item Frontend Framework : React.js (JavaScript)
    \item State Management : React Context API
    \item Communication with Backend : FastAPI
    \item Hosting \& Deployment : Azure Static Web Apps
 
\end{itemize}

\paragraph{component-based architecture}
\begin{enumerate}
    \item Navigation \& Routing \\
    Users can easily navigate between different sections of application, such as viewing product lists, accessing product details, and managing their shopping cart. React Router enables seamless client-side navigation, keeping your app as a single-page application (SPA).
    \begin{itemize}
        \item React Router Setup 
        \begin{itemize}
            \item React Router (react-router-dom) to handle the routing of different components, allowing users to navigate through pages like the homepage (/), product detail pages (/product/:id), and potentially a shopping cart or checkout page.
            \item For product detail pages, you're using dynamic routes with product/:id to fetch and display specific product data based on the product’s unique id. For instance:
            \item When a user clicks on a product in the catalog, the URL changes to something like /product/123, and the ProductDetail component is rendered with data for product 123.
            \item This is achieved using useNavigate and useParams hooks provided by React Router to capture the dynamic part of the URL.
        \end{itemize}
        \ 
        \item Navigation Links
        \begin{itemize}
            \item On the homepage (Home component), displaying a list of products. Each product has an image and name that users can click. When clicked, the app navigates to the product detail page using the navigate(/product/\${prod.id}) function.
        \end{itemize}
            \item API Communication \\
            API communication in app is responsible for sending and receiving data from the backend server. This includes fetching product data to populate catalog, handling cart actions (adding/removing items), and processing payments during checkout.
        \item Data Fetching
    \end{itemize}
 
\end{enumerate}
\paragraph{Key Features of the UI}
    \begin{enumerate}
        \item Product Catalog
        \item Product Search and Filtering
        \item Product Details Page
        \item Shopping Cart
        \item Checkout Process
    \end{enumerate}
\subsubsection{Application-Tier (Backend) - Business Logic}
\paragraph{Technology Stack}
\paragraph{Product Management}
\paragraph{Order Management}
\paragraph{Payment Processing}
\paragraph{Inventery Management}
\paragraph{Email Notification}

\subsubsection{Data-Tier (Database) - Databases}
\paragraph{Technology Stack}
\paragraph{Data bases}
\begin{enumerate}
    \item Product Data
    \begin{itemize}
        \item Product
        \item Catagories
        \end{itemize}
    \item Order Data
    \begin{itemize}
        \item Order 
        \item Order Details
        \end{itemize}
    \item Inventory Data
    \begin{itemize}
        \item Stocks 
        \item suppliers
        \end{itemize}
 
\end{enumerate}
\subsection{Technology Stack}
\begin{itemize}
    \item Frontend: 
    \item Backend: 
    \item Database: 

\end{itemize}

\subsection{System Diagrams}

\section{Implementation Process}
\subsection{Cloud Environment Setup}
Step-by-step account configuration and resource provisioning

\subsection{Service Integration}
\begin{itemize}
    \item Azure Load Balancer configuration
    \item Database replication setup
    \item Blob storage integration patterns
\end{itemize}

\subsection{Development Challenges}
\begin{itemize}
    \item State management in scaled environments
    \item Database connection pooling
    \item CSP-specific limitations encountered
\end{itemize}

\section{Operational Characteristics}
\subsection{Performance Metrics}

\subsubsection{Functional Test Cases}

\begin{longtable}{|c|p{4.5cm}|p{4.5cm}|c|}
\hline
\textbf{Test Case ID} & \textbf{Test Scenario} & \textbf{Expected Outcome} & \textbf{Status (Pass/Fail)} \\ \hline
\endfirsthead
\hline
\textbf{Test Case ID} & \textbf{Test Scenario} & \textbf{Expected Outcome} & \textbf{Status (Pass/Fail)} \\ \hline
\endhead
TC-001 & Load homepage and verify product listing & Homepage loads with products displayed correctly. & \\ \hline
TC-002 & Apply price filter (Ascending/Descending) & Products reorder correctly based on selected price. & \\ \hline
TC-003 & Apply "Out of Stock" filter & Only out-of-stock items are displayed. & \\ \hline
TC-004 & Apply "Fast Delivery" filter & Only products eligible for fast delivery show up. & \\ \hline
TC-005 & Filter by category & Products are filtered correctly by selected category. & \\ \hline
TC-006 & Search product by name or category & Products matching search are shown correctly. & \\ \hline
TC-007 & Clear filter functionality & All filters are removed, showing the full product list. & \\ \hline
TC-008 & Product Detail Page - Click Product & Clicking a product opens its detailed page. & \\ \hline
TC-009 & Product Detail Page - Load Product Details & Product details (name, description, price) are displayed. & \\ \hline
TC-010 & Add a product to cart & Product appears in cart with correct details. & \\ \hline
TC-011 & Increase product quantity in cart & Quantity updates and is reflected in the cart. & \\ \hline
TC-012 & Remove product from cart & Product is removed from the cart immediately. & \\ \hline
TC-013 & Proceed to checkout & Checkout page loads with the correct order summary. & \\ \hline
TC-014 & Select payment method - Stripe & Stripe payment option is selected and processed. & \\ \hline
TC-015 & Select payment method - PayPal & PayPal payment option is selected and processed. & \\ \hline
TC-016 & Complete order processing & Order confirmation message is displayed. & \\ \hline
TC-017 & Order confirmation email is received & Email is sent after order is placed. & \\ \hline
TC-018 & Shipment notification email is received & Email is sent when the order is shipped. & \\ \hline
TC-019 & Toggle Dark/Light Theme & Application switches between themes successfully. & \\ \hline
\end{longtable}

\subsection{Security Considerations}
\begin{itemize}
    \item Network security groups configuration
    \item Database encryption implementation
    \item Access control mechanisms
\end{itemize}

\section{Critical Analysis}
\subsection{Cloud Service Evaluation}
Cost-benefit analysis of selected Azure services

\subsection{Architectural Decisions}
Trade-off discussion between containerized vs serverless approaches

\section{Repository Documentation}
\subsection{GitHub Structure}
\begin{itemize}
    \item \textbf{Branching Strategy}  
    The project follows a simple main branch strategy where all contributors work directly on the main branch. 
    This ensures seamless integration without managing multiple branches.All changes should be committed with meaningful messages.  
    Code should be reviewed and tested before pushing to the main branch.

    \item \textbf{CI/CD Pipeline Configuration}  
    Continuous Integration and Continuous Deployment (CI/CD) is implemented to automate the software development lifecycle, ensuring that code changes are tested, built, and deployed efficiently.  

    \begin{itemize}
        \item **Continuous Integration (CI)**:  
        Every code commit triggers an automated build and testing process. This includes:
        \begin{itemize}
            \item Running unit tests to validate individual components.
            \item Performing integration tests to ensure compatibility between different modules.
            \item Static code analysis for linting and security vulnerabilities.
        \end{itemize}

        \item **Continuous Deployment (CD)**:  
        Once the CI stage passes successfully, the application is automatically deployed to the appropriate environment. This process includes:
        \begin{itemize}
            \item Deploying to a staging environment for final testing.
            \item Running automated acceptance tests before production deployment.
            \item Deploying to production with rollback mechanisms in case of failure.
        \end{itemize}
        
    \item Documentation standards
\end{itemize}
\end{itemize}

\subsection{Contribution Tracking}
Commit history analysis and individual contribution breakdown

\section{Conclusion}
\subsection{Project Outcomes}
Summary of achieved objectives and demo capabilities

\subsection{Future Enhancements}
Potential improvements for production readiness

\begin{thebibliography}{9}

\end{thebibliography}
\end{document}
